\newpage
\section{Results}

This section presents the results and their respective conclusion on all the experiments and their preliminary tests.

\subsection{Girvan-Newman aproximation}


\textbf{Preliminary test}

The results of the semi-supervised classification task on the CORA dataset are show on the following table:



\begin{tabular}{llrlr}
\toprule
    Model &           params &   Val Loss &   Test Accuracy &   Duration \\\\
\midrule
   SGConv &   100\_epochs=200 &     1.7587 &   0.790 ± 0.018 &      0.585 \\
 ChebConv &   100\_epochs=200 &     0.8133 &   0.769 ± 0.027 &      5.049 \\
    APPNP &   100\_epochs=200 &     0.8894 &   0.808 ± 0.017 &      0.984 \\
  GATConv &   100\_epochs=200 &     0.8111 &   0.803 ± 0.016 &      1.967 \\
  GCNConv &   100\_epochs=200 &     0.8864 &   0.789 ± 0.018 &      1.091 \\
\bottomrule

\end{tabular}




\textbf{Main experiment}

As detailed in the experiments section, the main experiment is to train a model to approximate the edge betweenness of part of the edges of a graph, in a semi-supervised manner. The results, measured with the Normalised Root Mean Square Error (nrmse) are shown in the folling table:



A complementary test on the node betweenness is shown in the next table:



A complementary test on the PageRank aproximation is shon in the table \ref{}



\texbf{Conclusion}
% conclusion on results

A model with a good performance has not been found. All the models exhibit underfit or a bias that makes them poor aproximations.
Spatial based graph convolutional networks have the problem of losing site of long range interactions. However, the shortest path is long range by nature.
Spectral based methods are not affected by this limitation, but the aproximated spectral method from ChebNets is, eve thought we tried to minimze it.



% 2) Function Renaming:
% results table: noisy dataset:   models | cv score | test score macro avg
% results table: v1 dataset:   models | cv score | test score macro avg
% results table: v2 dataset:   models | cv score | test score macro avg
% results table: v3 dataset:   models | cv score | test score macro avg

% conclusion on results



% idea: polyorphism on code, hacking the art of exploitation chap5, even with xor or sub, a code can be rewritten => STRUCTURE IS NOT KEY TO FUNCTIONALITY ON CODE (but still I was hoping on some similar structure coming from similar compilers/programing languages/SO interfaces..)
